\documentclass{article}
\usepackage{amsmath, amssymb}
\usepackage{hyperref}

\title{Different versions that were used}
\author{}


\begin{document}
\maketitle

Here are some of the versions of different software that we used. More detailed
description can be found in few other text files in the repository.

\section{Conda/2022-09-08 on Polaris}

This module was used in PointNet-Atlas and FFN Runs on Polaris. The versions
of Tensorflow and PyTorch are below:

\subsection{TensorFlow}
\begin{enumerate}
\item tensorflow                2.10.0                   
\item tensorflow-addons         0.17.1                   
\item tensorflow-datasets       4.6.0                    
\item tensorflow-estimator      2.10.0                   
\item tensorflow-io-gcs-filesystem 0.27.0               
\item tensorflow-metadata       1.10.0                   
\item tensorflow-probability    0.17.0
\end{enumerate}

\subsection{PyTorch}
\begin{enumerate}
\item torch                     1.12.0a0+git664058f          
\item torch-geometric           2.1.0.post1              
\item torch-scatter             2.0.9                    
\item torch-sparse              0.6.15                   
\item torch-tb-profiler         0.4.0                    
\item torchinfo                 1.7.0                    
\item torchmetrics              0.9.3                    
\item torchvision               0.13.0a0+da3794e          
\item torchviz                  0.0.2                    
\end{enumerate}

\subsection{Auxiliary}
\begin{enumerate}
    \item cupy-cuda116              10.6.0
    \item h5py                      3.7.0
    \item horovod                   0.25.0
    \item hydra-colorlog            1.2.0
    \item hydra-core                1.2.0
    \item jax                       0.3.17
    \item jaxlib                    0.3.15+cuda11.cudnn82
    \item keras                     2.10.0
    \item keras-applications        1.0.8
    \item keras-preprocessing       1.1.2
    \item mpi4jax                   0.3.10.post1
    \item mpi4py                    3.1.3

\end{enumerate}

\section{JLSE Nodes for AMD MI250}
It is a little not so straightforward to run on MI250 on JLSE nodes. I had to
install things on a separate environment 
\verb|/home/hossainm/miniconda3/envs/dist-ct-tf-amd-aux-2|.
Here are the versions:

\subsection{TensorFlow}

\begin{enumerate}
\item tensorflow-estimator      2.12.0
\item tensorflow-io-gcs-filesystem 0.32.0
\item tensorflow-rocm           2.12.0.560
\end{enumerate}

and these were used with another pre-built module rocm/5.5.0 and 
openmpi/4.1.1-llvm compiler.


\subsection{PyTorch}
PyTorch worked out of the box with one of the pre-built modules. That was 
\verb|conda/amd/2022-01-11|.

\begin{enumerate}
\item torch                     1.13.0+rocm5.2
\item torchaudio                0.13.0+rocm5.2           
\item torchvision               0.14.0+rocm5.2 
\end{enumerate}

But, these were used with another pre-built module rocm/5.5.0 and 
openmpi/4.1.1-llvm compiler.

\section{Frameworks/2023.05.15.001 on Sunspot}
We used this framework module on Sunspot. The versions here are:

\subsection{TensorFlow}

\begin{enumerate}
\item tensorflow                2.12.0
\item tensorflow-estimator      2.12.0
\item tensorflow-io-gcs-filesystem 0.32.0
\end{enumerate}

\subsection{PyTorch}

\begin{enumerate}
\item torch                     1.13.0a0+git6c9b55e
\item torchvision               0.14.1a0+5e8e2f1
\end{enumerate}




\end{document}
